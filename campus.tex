\section{Campus}

\subsection{Interfaz}

\sexc{CAMPUS}
$\textbf{usa}$  
\generos{campus}

%%%%%%%%%%%%%%%%%%%%%%%%%%%%%%%%%%%%%%%%%%%%%%%%%%%%%%%%%%%%%%%%%%%%%%%%%%%%

\subsubsection*{Operaciones}


\operacion{ArmarCampus}{in ancho: nat, alto: nat}{res : campus}
 {true}
 {$\var{res} \igobs crearCampus(ancho, alto) $}
 {Crea el campus, sin obstáculos}
 {O(ancho x alto)}
 {}
 
 \operacion{agregarObs}{in/out c: campus, in p: pos}{}
 {$ \var(c) \equiv \var(c_0) $}
 {$\var{c} \igobs agregarObstaculo(p, c_0) $}
 {Agrega un obstáculo al campus}
 {O(1)}
 {}
 
  \operacion{Alto}{in c: campus}{res : nat}
 {true}
 {$\var{res} \equiv alto(c) $}
 {Indica la cantidad de filas de c}
 {O(1)}
 {}

  \operacion{Ancho}{in c: campus}{res : nat}
 {true}
 {$\var{res} \equiv alto(c) $}
 {Indica la cantidad de columnas de c}
 {O(1)}
 {}
  
  \operacion{Ocupada}{in c: campus, p: pos}{res : bool}
 {PosValida(c,p)}
 {$\var{res} \iff \pi_1(grilla(c)[\pi_1(p)][\pi_2(p)])  $}
 {Comprueba si una posición está ocupada}
 {O(1)}
 {}
   
  \operacion{PosValida}{in c: campus, p: pos}{res : bool}
 {true}
 {$\var{res} \iff (\pi_1(p)<ancho(c)\land \pi_2(p)<alto(c)) $}
 {Comprueba que una posición exista dentro del campus.}
 {O(1)}
 {}
    
  \operacion{EsIngreso}{in c: campus, p: pos}{res : bool}
 {PosValida(c,p)}
 {$\var{res} \iff (\pi_2(p)=alto(c)-1 \lor \pi_2(p)=0) $}
 {Comprueba si una posición es un ingreso al campus.}
 {O(1)}
 {}
   
  \operacion{IngresoSup}{in c: campus, p: pos}{res : bool}
 {PosValida(c,p)}
 {$\var{res} \iff \pi_2(p)=0 $}
 {Comprueba si una posición es un ingreso superior al campus.}
 {O(1)}
 {}
 
 \operacion{IngresoInf}{in c: campus, p: pos}{res : bool}
 {PosValida(c,p)}
 {$\var{res} \iff \pi_2(p)=alto(c)-1 $}
 {Comprueba si una posición es un ingreso superior al campus.}
 {O(1)}
 {}

 \operacion{Distancia}{in c: campus, p1: pos, p2: pos}{res : nat}
 {$PosValida(c,p1) \land PosValida(c,p2)$}
 {$\var{res} \equiv distancia (p1, p2, c) $}
 {Comprueba si una posición es un ingreso inferior al campus.}
 {O(1)}
 {}
 
 \operacion{Vecinos}{in c: campus, p: pos}{res : conj(pos)}
 {$PosValida(c,p)$}
 {$\var{res} \equiv vecinos (p, c) $}
 {devuekve el conjunto de vecinos de una posición.}
 {O(1)}
 {}
 
Las complejidades están en función de las siguientes variables:\\
$al$ : cantidad de filas del campus, \\
$an$ : cantidad de columnas del campus, \\
$k$ : la cola de paquetes más larga de todas las computadoras. 
\\ \\



%%%%%%%%%%%%%%%%%%%%%%%%%%%%%%%%%%%%%%%%%%%%%%%%%%%%%%%%%%%%%%%%%%%%%%%%%%%%%
\subsection{Representación}

\serc{estr}{
	\\
	\donde{estr}{\tupla{ ancho: \mbox{nat}, alto: \mbox{nat}, 
	 grilla: \mbox {arreglo(arreglo(\tupla{Ocupado: bool, EsObst: bool, EsAgente: bool, \tupla{pl:nat, nombre:string}}))}}

}	
}

\subsubsection*{Invariante de representación}

\begin{enumerate}
  \item Todos los IP de \textit{compus} pertenecen al conjunto de claves de \textit{CompusPorPref} y la longitud de dicho arreglo 
		es igual al cardinal de las claves del diccionario. 
  \item Los pN de las tuplas que tiene el arreglo \textit{compus} apuntan al conjunto de paquetes(PorNom) 
		de un significado en \textit{CompusPorPref} cuya clave es igual al IP de esa posición en el arreglo.
  \item Los pN' apuntan al conjuno de paquetes(porNom') de un significado en \textit{CompusPorPref} cuya clave es igual al IP 
		de esa posición en el arreglo
  \item Los paquetes del significado pN' son iguales a los paquetes de pN
  \item El origen de pN' es distinto al destino de pN' y ambos son posiciones válidas del arreglo compus
  \item PosActual de pN' es una posicion válida del arreglo compus
  \item La $\#PaquetesEnviados$ de cada compu es mayor o igual a la actual cantidad total de paquetes que pasaron por esa compu
  \item Todos los conjuntos de los significados de \textit{CompusPorPref} son disjuntos dos a dos.
  \item Los conjuntos de los campos de la tupla PorNom, PorPrior son iguales.
  \item La matriz de caminosMinimos es cuadrada de lado n, con n igual al tamaño del arreglo de compus.
  \item Para cualquier compu en el sistema f,d caminosMinimos[f][d] se corresponde con caminoMinimo(red,f,d)
  \item La longitud de \textit{CaminosMinimos} es igual a la longitud del arreglo que tiene \textit{CaminosMinimos} en cada posición.
  \item La longitud del arreglo, que tiene un arreglo de \textit{CaminosMinimos} es menor o igual a la longitud de \textit{CaminosMinimos}.
  \item Los elementos del arreglo anteriormente mencionado son IPs del diccionario \textit{CompusPorPref} y no tiene repetidos.
  \item La computadora que más paquetes envió es aquella cuyo índice es igual a \textit{LaQMasEnvio}
\end{enumerate}

\rep{sistema}{s}  \\
 $1.\ \paratodo{String}{s}\ \fun{$def?$}(s, s.CompusPorPref), (\exists c : compu), \fun{$esta?$}(c,s.Compus)\ \land\ \pi_1(c) = s\ \land\ \fun{longitud}(s.Compus) = \#\Fun{claves}(s.CompusPorPref)$\\
 $2.\ \paratodo{compu}{c}\ \fun{esta?}(c, s.Compus), *\pi_2(c) = \fun{obtener}(\pi_1(c),s.CompusPorPref) $\\
 $3.\ \paratodo{compu}{c}\ \fun{esta?}(c, s.Compus), *\pi_3(c) = \fun{obtener}(\pi_3(c),s.CompusPorPref) $\\
 $4,5,6.$ \\$\ (\paratodo{nat}{c})\ 0 \le c < Longitud(s.compus) \impluego$\\
  \indent $\ \ Longitud(s.compus[c].pN) = Longitud(s.compus[c].pN')$ $\land$ \\
  \indent $\ \ (\paratodo{paquetePos}{p}) \fun{esta?}(p,s.compus[c].pN') \impluego$\\
  \indent \indent \indent $\ \ \ \ \fun{esta}(\pi_1(p), s.compus[c].pN)$ $\land$ $0 \le indiceOrigen(p) < Longitud(s.compus)$\\
  \indent \indent \indent $\ \ \ \ \land$ $0 \le indiceDestino(p) < Longitud(s.compus)$\\
  \indent \indent \indent $\ \ \ \ \land$ $\ 0 \le posActual(p) < Longitud(s.compus)$\\
  \indent \indent \indent $\ \ \ \ \land$ $\ \neg(indiceDestino(p)=indiceOrigen(p))$\\
 $7.\ \ (\paratodo{nat}{c})$ $0 \le c < Longitud(s.compus)$ $\impluego$\\
 \indent \indent $\ \ (\paratodo{paquetePos}{p})$ $\fun{pertenece}(s.compus[c].pN', p)$ $\impluego$\\
 \indent \indent \indent $\ \ \ \ \beta(\fun{esta}(s.compus[c],caminoMinimo(s.red,s.compus[indiceOrigen(p)],s.compus[posActual(p)])))$\\
 $8.\ \ \paratodo{String}{s,t}\ \fun{def?}(s, s.CompusPorPref)\ \land\ \fun{def?}(t, s.CompusPorPref)\ \land\ s \neq t \impluego \\ \fun{obtener}(s, s.CompusPorPref)\ \bigcap\ \fun{obtener}(t, s.CompusPorPref) = \emptyset$\\
 $9.\ \paratodo{String}{s}\ \fun{def?}(s, s.CompusPorPref)\ \impluego\ \pi_1(\fun{obtener}(s, s.CompusPorPref)) =\\ \pi_2(\fun{obtener}(s, s.CompusPorPref))$ \\
 $10.\ Longitud(s.compus) = Longitud(CaminosMinimos(s))$ $\land$\\
 \indent \indent \indent \indent$\ \ \ (\paratodo{nat}{i})$ $0 \le i < Longitud(s.compus)$ $\impluego$\\
 \indent \indent \indent \indent$\ \ \ \ \ Longitud(s.CaminosMinimos[i]) = Longitud(s.compus)$\\
 $11.\ (\paratodo{nat}{f,d})$ $\neg(f=d) \land$ $0 \le f,d < Longitud(s.compus)$ $\impluego$\\
 \indent \indent $\ \ \ CaminosMinimos[f][d]$ $=$\\
 \indent \indent $\ \ \ caminoMinimo(s.red,ipACompu(s.red,\pi_1(s.compus[f])),ipACompu(s.red,\pi_1(s.compus[d])))$\\
 $12,13,14.\ (\paratodo{nat}{i,j}),\ 0 \leq i,j < \fun{longitud}(s.CaminosMinimos) \impluego \fun{longitud}(s.CaminosMinimos) = \fun{longitud}(s.CaminosMinimos[i])\ \land\ \fun{longitud}(s.CaminosMinimos[i][j]) < \fun{longitud}(s.CaminosMinimos)\ \land\ (\paratodo{nat}{e}), \fun{esta?}(e, s.CaminosMinimos[i][j]) \Rightarrow \fun{pertenece}(e,s.CompusPorPref)  $\\
 $15. \paratodo{compu}{c}\ \fun{esta?}(c, s.Compus)\ \impluego\ \pi_3(c) \leq \pi_3(s.Compus[s.LaQMasEnvio])$
%\subsubsection*{Función de abstracción}
 %\abs{sistema}{sist}{dcnet}{s} \\
%$(\fun{s.} = \NULL \land \fun{*r.der} = \NULL \ssi nil?(a)) \ \oluego$ \\

% $\fun{*r.izq} = izq(a) \land \fun{*r.der} = der(a) \ \land$
%$r.val = raiz(a) $

%%%%%%%%%%%%%%%%%%%%%%%%%%%%%%%%%%%%%%%%%%%%%%%%%%%%%%%%%%%%%%%%%%%%%%%%%%%%%%%%%%


\subsubsection*{Funci\'on de abstracción}

\abs{dcnet}{DCNet}{s}{dc} \\
$red(dc)=$*$(s.red) \land (\forall c:compu, c\in compus(dc))( enEspera(dc,c)=$*$(enEspera(s,c)) \land$ \\
$  cantidadEnviados(dc,c)=cantidadEnviados(s,c)) \land $ \\ 
$(\forall p:paquete,paqueteEnTransito?(dc,p)) caminoRecorrido(dc,p)=$*$(caminoRecorrido(s,p)) $

\newpage

\subsection{Algoritmos}

\algoritmo{icrearSistema}{in r:red}{res:dcnet}{
%inicializacion Caminos Minimos
	\State $res.red \larr r$
	\State $n \larr Longitud(\Fun{compus}(red))$ \complejidad{O(1)}
	\State $i\larr 0$
	\State $j\larr 0$
		\complejidad{O(1)}
	\State $res.Compus \larr \Fun{CrearArreglo}(n)$
		\complejidad{O(n)}
	\State $res.CaminosMinimos\larr \Fun{CrearArreglo}(n)$
		\complejidad{O(n)}
	\Var{p: arreglo$\_$dimensionable de puntero(conjLog(paquete))}	
	\While{i$<$n} \complejidad{O($L*n^5$)}
		\complejidad{O(n)}
		\State $res.CaminosMinimos[i] \larr \Fun{CrearArreglo}(n)$ 
			\complejidad{O(n)}
		\State s:$ \ <nat,conjLog(paquete,<_{id}),conjLog(paquete,<_{p}), $\\$ conjLog(paquetePos,<_{id}),conjLog(paquetePos,<_{p}) >$
		%HAY que ver bien donde definir la relacion!!
		\State $\pi_1(s) \larr compu(r,i)$
		\State $\pi_2(s) \larr \Fun{nuevo}()$
		\State $\pi_3(s) \larr \Fun{nuevo}()$
		\State $\pi_4(s) \larr \Fun{nuevo}()$
		\State $\pi_5(s) \larr \Fun{nuevo}()$
		\State \Fun{definir}(res.CompusPorPref,compu(r,i),s)
			\complejidad{O(L)}
		\State $p[i] \larr \pi_3(s)$
		\State $p'[i] \larr \pi_5(s) $
		\State $res.Compus[i] \larr $ $<compu(r,i),p[i],p'[i],0>$ 
			\complejidad{O(1)}
		\While{j$<$n} \complejidad{O($L*n^4$)}
			\complejidad{O(n)}
			\State $res.CaminosMinimos[i][j] \larr caminoMinimo(compu(r,i),compu(r,j),r)$
				\complejidad{O($L*n^3$)}
			\State $j++$
		\EndWhile
		\State $i++$	
	\EndWhile
	\State $res.LaQMasEnvio \larr 0$
		\complejidad{O(1)}
%holi
}
{O($L \times n^5$)}

\algoritmo{icrearPaquete}{in/out s:dcnet,in/out p: paquete}{}{ %hay que cambiar nombres segun heap y avl
  \State $t_1:\ <nat,conjLog(paquete,<_{id}),conjLog(paquete,<_{p}), $\\$ conjLog(paquetePos,<_{id}),conjLog(paquetePos,<_{p}) >$
  \State $t_1 \larr \Fun{Obtener}(origen(p),\ \var{s}.CompusPorPref)$
        \complejidad{O($L$)}
  \State $t_2:\ <nat,conjLog(paquete,<_{id}),conjLog(paquete,<_{p}), $\\$ conjLog(paquetePos,<_{id}),conjLog(paquetePos,<_{p}) >$
  \State $t_2 \larr \Fun{Obtener}(destino(p),\ \var{s}.CompusPorPref)$
  	\complejidad{O($L$)}
  \State $p$'$:paquetePos$ 
  \State $\Fun{indiceOrigen}(p$'$) \larr \pi_1(t_1)$
  	\complejidad{O(1)}
  \State $\Fun{indiceDestino}(p$'$) \larr \pi_1(t_2)$
  	\complejidad{O(1)}
  \State $\Fun{posActual}(p$'$) \larr 0$
  \State \Fun{insertar}($\pi_2(t),p$) %agrega segun nombre
        \complejidad{O($log(k)$)}
  \State \Fun{insertar}($\pi_3(t),p$) %agrega segun prioridad
        \complejidad{O($log(k)$)}
  \State \Fun{insertar}($\pi_4(t),p$') %agrega segun nombre
        \complejidad{O($log(k)$)}
  \State \Fun{insertar}($\pi_5(t),p$') %agrega segun prioridad
        \complejidad{O($log(k)$)}
}
{O($L + log(k)$)}

%%%%%%%%%%%%%%%%%%%%%%%%%%%%%%%%%%%%%%%%%%%%%%%%%%%%%%%%%%%%%%%%%%%%%%%%%%%%%%%%%

\subsection{Servicios Usados}
Del modulo ConjLog requerimos \fun{pertenece}, \fun{buscar}, \fun{menor}, \fun{insertar} y \fun{borrar} en O($log(k)$) .

Del modulo Diccionario Por Prefijos requerimos \fun{Def?}, \fun{obtener} en O($L$).
