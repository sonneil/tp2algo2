\section{Campus}

\subsection{Interfaz}

\sexc{CAMPUS}
$\textbf{usa}$  
\generos{campus}

%%%%%%%%%%%%%%%%%%%%%%%%%%%%%%%%%%%%%%%%%%%%%%%%%%%%%%%%%%%%%%%%%%%%%%%%%%%%

\subsubsection*{Operaciones}


\operacion{ArmarCampus}{in ancho: nat, alto: nat}{res : campus}
 {true}
 {$\var{res} \igobs crearCampus(ancho, alto) $}
 {Crea el campus, sin obstáculos}
 {O(ancho x alto)}
 {}
 
 \operacion{agregarObs}{in/out c: campus, in p: pos}{}
 {$ \var(c) \equiv \var(c_0) $}
 {$\var{c} \igobs agregarObstaculo(p, c_0) $}
 {Agrega un obstáculo al campus}
 {O(1)}
 {}
 
  \operacion{Alto}{in c: campus}{res : nat}
 {true}
 {$\var{res} \equiv alto(c) $}
 {Indica la cantidad de filas de c}
 {O(1)}
 {}

  \operacion{Ancho}{in c: campus}{res : nat}
 {true}
 {$\var{res} \equiv alto(c) $}
 {Indica la cantidad de columnas de c}
 {O(1)}
 {}
  
  \operacion{Ocupada}{in c: campus, p: pos}{res : bool}
 {PosValida(c,p)}
 {$\var{res} \iff \pi_1(grilla(c)[\pi_1(p)][\pi_2(p)])  $}
 {Comprueba si una posición está ocupada}
 {O(1)}
 {}
   
  \operacion{PosValida}{in c: campus, p: pos}{res : bool}
 {true}
 {$\var{res} \iff (\pi_1(p)<ancho(c)\land \pi_2(p)<alto(c)) $}
 {Comprueba que una posición exista dentro del campus.}
 {O(1)}
 {}
    
  \operacion{EsIngreso}{in c: campus, p: pos}{res : bool}
 {PosValida(c,p)}
 {$\var{res} \iff (\pi_2(p)=alto(c)-1 \lor \pi_2(p)=0) $}
 {Comprueba si una posición es un ingreso al campus.}
 {O(1)}
 {}
   
  \operacion{IngresoSup}{in c: campus, p: pos}{res : bool}
 {PosValida(c,p)}
 {$\var{res} \iff \pi_2(p)=0 $}
 {Comprueba si una posición es un ingreso superior al campus.}
 {O(1)}
 {}
 
 \operacion{IngresoInf}{in c: campus, p: pos}{res : bool}
 {PosValida(c,p)}
 {$\var{res} \iff \pi_2(p)=alto(c)-1 $}
 {Comprueba si una posición es un ingreso superior al campus.}
 {O(1)}
 {}

 \operacion{Distancia}{in c: campus, p1: pos, p2: pos}{res : nat}
 {$PosValida(c,p1) \land PosValida(c,p2)$}
 {$\var{res} \equiv distancia (p1, p2, c) $}
 {Comprueba si una posición es un ingreso inferior al campus.}
 {O(1)}
 {}
 
 \operacion{Vecinos}{in c: campus, p: pos}{res : conj(pos)}
 {$PosValida(c,p)$}
 {$\var{res} \equiv vecinos (p, c) $}
 {Devuelve el conjunto de vecinos de una posición.}
 {O(1)}
 {}
  \operacion{aPosMasCercana}{in c: campus, p: pos, con:conj(pos)}{res : pos}
 {$PosValida(c,p)\land #con>0$}
 {$(\forall p_1) PosValida(c, p_1) \impluego Distancia(c,p) \leq Distancia(c, p_1) $}
 {Dado un conjunto de posiciones y una posiciones, da la más cercana a la posicion dada.}
 {O($#con$)}
 {}
 
Las complejidades están en función de las siguientes variables:\\
$al$ : cantidad de filas del campus, \\
$an$ : cantidad de columnas del campus, \\
$k$ : la cola de paquetes más larga de todas las computadoras. 
\\ \\



%%%%%%%%%%%%%%%%%%%%%%%%%%%%%%%%%%%%%%%%%%%%%%%%%%%%%%%%%%%%%%%%%%%%%%%%%%%%%
\subsection{Representación}

\serc{estr}{
	\\
	\donde{estr}{\tupla{ ancho: nat, alto: nat,	grilla: \mbox {arreglo(arreglo(\tupla{Ocupado: bool, EsObst: bool, EsAgente: bool, HiippieoEst:\tupla{pl:nat, nombre:string} }))}}
  }	
}

\subsubsection*{Invariante de representación}

\begin{enumerate}
  \item  El tamaño de todos los arreglos internos de la grilla es el mismo e igual al alto del campus
  \item El tamaño del arreglo principal de la grilla es igual al ancho del campus
  

\rep{campus}{}  \\
 $|c.grilla|=c.ancho \land (\forall n:nat, n \in \lbrack 0, c.ancho)) |c.grilla [n]|=c.alto$
%\subsubsection*{Función de abstracción}
 %\abs{sistema}{sist}{dcnet}{s} \\
%$(\fun{s.} = \NULL \land \fun{*r.der} = \NULL \ssi nil?(a)) \ \oluego$ \\

% $\fun{*r.izq} = izq(a) \land \fun{*r.der} = der(a) \ \land$
%$r.val = raiz(a) $

%%%%%%%%%%%%%%%%%%%%%%%%%%%%%%%%%%%%%%%%%%%%%%%%%%%%%%%%%%%%%%%%%%%%%%%%%%%%%%%%%%


\subsubsection*{Funci\'on de abstracción}

\abs{campus}{Campus}{c}{cE} \\
$columnas(cE)=c.ancho \land filas(cE)=c.alto\land (\forall p:pos, PosValida?(cE,p)) Ocupada?(cE,p)=Ocupada?(c,p)$

\newpage

\subsection{Algoritmos}

\algoritmo{iArmarCampus}{in ancho:nat, in alto:nat}{res:campus}{
%inicializacion Caminos Minimos
	\State $res.ancho \larr ancho$ \complejidad {O(1)}
	\State $res.alto \larr alto$ \complejidad {O(1)}
	\State $res.grilla \larr CrearArreglo(ancho)$ \complejidad{O(an)}
	\State $i\larr 0$ \complejidad {O(1)}
			\complejidad{O(1)}
	\While{$i<ancho$} \complejidad{O($al*an$)}
		\complejidad{O(an)}
		\State $res.grilla[i] \larr \Fun{CrearArreglo}(alto)$ 
			\complejidad{O(al)}
		\State $j \larr 0$ \complejidad {O(1)}
		\While {$j<alto$} \complejidad {O (al)}
			\State $res.grilla [i][j] \larr <False,False,False,<0,"">>$ \complejidad {O(1)}
		\EndWhile
	\EndWhile
	
%holi
}
{O($an*al$)}

\algoritmo{iAgregarObs}{in/out c:campus,in p: pos}{}{ %hay que cambiar nombres segun heap y avl
  \State $\pi_1(c.grilla[p.X][p.Y]) \larr True$
  	\complejidad {O(1)}
  \State $\pi_2(c.grilla[p.X][p.Y]) \larr True$ \complejidad {O(1)}
}
{O(1)}
\algoritmo{iAlto}{in c:campus}{res:nat}{
	\State $res \larr c.alto$ \complejidad {O(1)}
}
{O(1)}
\algoritmo{iAncho}{in c:campus}{res:nat}{
	\State $res \larr c.ancho$ \complejidad {O(1)}
}
{O(1)}
\algoritmo{iOcupada}{in c:campus, in p:pos}{res:bool}{
	\State $res \larr c.grilla[p.X][p.Y].Ocupado$\complejidad {O(1)}
}
{O(1)}
\algoritmo{iPosValida}{in c:campus, in p:pos}{res:bool}{
	\State $res \larr p.X<c.ancho \land p.Y<c.alto$  \complejidad {O(1)}
}
{O(1)}
\algoritmo{iEsIngreso}{in c:campus, in p:pos}{res:bool}{
	\State $res \larr Y.p=0 \lor Y.p=c.alto-1 $\complejidad {O(1)}
}
{O(1)}
\algoritmo{iIngresoSup}{in c:campus, in p:pos}{res:bool}{
	\State $res \larr Y.p=0$\complejidad {O(1)}
}
{O(1)}
\algoritmo{iIngresoInf}{in c:campus, in p:pos}{res:bool}{
	\State $res \larr Y.p=c.alto-1$\complejidad {O(1)}
}
{O(1)}
\algoritmo{iDistancia}{in c:campus, in p1:pos,p2:pos}{res:nat}{
	\State $resX \larr 0$
	\State $resY \larr 0$
	\If{$p1.X<p2.X$}
		\State $resX \larr p2.X-p1.X$
	\Else
		\State $resX \larr p1.X-p2.X$
	\Endif
	\If{$p1.Y<p2.Y$}
		\State $resY \larr p2.Y-p1.Y$
	\Else
		\State $resy \larr p1.Y-p2.Y$
	\Endif
	\State $res \larr resX+resY$
}
{O(1)}
\algoritmo{iVecinos}{in c:campus, p:pos}{res:conj(pos)}{
	\State $res \larr Vacio() $
	\State $pn \larr <p.X,p.Y+1>$
	\IF {$PosValida(c,pn)$}
		\State agregar(res,pn)
	\Endif
	\State $pn \larr <p.X,p.Y-1>$
	\IF {$PosValida(c,pn)$}
		\State agregar(res,pn)
	\Endif
	\State $pn \larr <p.X+1,p.Y>$
	\IF {$PosValida(c,pn)$}
		\State agregar(res,pn)
	\Endif
	\State $pn \larr <p.X-1,p.Y>$
	\IF {$PosValida(c,pn)$}
		\State agregar(res,pn)
	\Endif
	
}
{O(1)}
\algoritmo{iaPosMasCercana}{in c:campus, in p:pos, con:conj(pos)}{res:pos}{
	\State $it \larr CrearIt(con)$
	\State $res \larr it.siguiente()$
	\While{HaySiguiente?(it))}
		\If{$Distancia(c,p,res)>Distancia(c,it.siguiente(),res)$}
			\State $res \larr it.siguiente()$
			\While{HayAnterior()}
				\State $EliminarAnterior(it)$
			\Endwhile
			\State Avanzar(it)
		\Else
			\If{$Distancia(c,p,res)<Distancia(c,it.siguiente(),res)$}
				\State $EliminarSiguiente(it)$
			\Else
				\State Avanzar(it)
			\Endif
		\Endif
	\Endwhile
	\State $it \larr CrearIt(con)$
	\State $res \larr it.siguiente()$
	\While{HaySiguiente?(it))}
		\If{$Distancia(c,<0,0>,p)>Distancia(c,it.siguiente(),<0,0>)$}
			\State $res \larr it.siguiente()$
			\While{HayAnterior()}
				\State $EliminarAnterior(it)$
			\Endwhile
			\State Avanzar(it)
		\Else
			\If{$Distancia(c,p,<0,0>)<Distancia(c,it.siguiente(),<0,0>)$}
				\State $EliminarSiguiente(it)$
			\Else
				\State Avanzar(it)
			\Endif
		\Endif
	\Endwhile	
	\State $it \larr CrearIt(con)$
	\State $res \larr it.siguiente()$
	\While{HaySiguiente?(it))}
		\If{$p.Y<it.siguiente().Y$}
			\State $res \larr it.siguiente()$
			\While{HayAnterior()}
				\State $EliminarAnterior(it)$
			\Endwhile
			\State Avanzar(it)
		\Else
			\If{$p.Y>it.siguiente().Y$}
				\State $EliminarSiguiente(it)$
			\Else
				\State Avanzar(it)
			\Endif
		\Endif
	\Endwhile
	
}
{O($#con$)}

%%%%%%%%%%%%%%%%%%%%%%%%%%%%%%%%%%%%%%%%%%%%%%%%%%%%%%%%%%%%%%%%%%%%%%%%%%%%%%%%%

\subsection{Servicios Usados}
Del modulo ConjLog requerimos \fun{pertenece}, \fun{buscar}, \fun{menor}, \fun{insertar} y \fun{borrar} en O($log(k)$) .

Del modulo Diccionario Por Prefijos requerimos \fun{Def?}, \fun{obtener} en O($L$).
