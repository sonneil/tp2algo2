%\\\\\\\\\\\\!TEX root = tp2.tex

\section{Diccionario Rapido}

Es un diccionario que dado una clave nat distribuida uniformemente, nos da su significado en promedio O(1)

\subsection{Interfaz}

  \textbf{par\'ametros formales}
  
  \textbf{g\'eneros} Nat, $\beta$\\

\sexc{Diccionario(Nat,conj($\beta$)), Iterador Bidireccional}
\generos{diccR(Nat,conj($\beta$))}

$\textbf{usa}$  
Bool, Nat, Conjunto($\beta$)


%%%%%%%%%%%%%%%%%%%%%%%%%%%%%%%%%%%%%%%%%%%%%%%%%%%%%%%%%%%%%%%%%%%%%%%%%%%%

\subsubsection*{Operaciones}


\operacion{crear}{in dicc:Dicc(c:nat s:\beta)}{res : diccSR(Nat,\beta)}
 {true}
 {$res \igobs Nuevo()$}
 {Crea un diccionario rapido.}
 {O(n)}
 {No hay aliasing}

\operacion{obtener}{in/out v : diccR(Nat;conj(\alpha)),in p : nat}{res : conj($\alpha$)}
 {Definido?(p,v)}
 {$Obtener(p,v)$}
 {Retorna el significado actual, para la clave dada.}
 {O(1)}
 {El retorno se hace por referencia, hay aliasing entre el objeto devuelto y el del contenedor}
 
Las complejidades est\'an en funci\'on de las siguientes variables:\\
$n$ : la cantidad total de claves definidas en el diccionario. \\
\\ \\


%%%%%%%%%%%%%%%%%%%%%%%%%%%%%%%%%%%%%%%%%%%%%%%%%%%%%%%%%%%%%%%%%%%%%%%%%%%%%
\subsection{Representaci\'on}

\serc{estr}{
	\\
	\donde{estr}{\tupla{
					accesoRapido : \mbox{vector(acceso:Dicc(nat,itDicc(nat,\beta)))},
					contenedor : \mbox{Dicc(c:nat,s:\beta)}
				}}
}
El contenedor esta implementado sobre Dicc lineal\\
Cada posicion del vector esta implementado sobre Dicc lineal

\subsubsection*{Invariante de representaci\'on}
\begin{enumerate}
  \item El diccionario que resulta de Definir ordenadamente los siguientes de cada (clave, sign) del dicc de cada posicion del vector, es igual al contenedor
  \item El tamaño del vector es igual al tamaño del contenedor
  \item La suma de todos los tamanios de los dicc en vector es igual al tamaño del contenedor
\end{enumerate}


\subsection{Algoritmos}

\algoritmo{iNuevo}{in diccACopiar:dicc(c:Nat, s:$\beta$)}{res:diccR()}{
	\State $it \larr CrearIt(diccACopiar)$ \complejidad{O(1)}
	\State // Inicializo el vector
	\While {$it.haySiguiente$}
		\State $res.accesoRapido.agregarAtras(NULL)$ \complejidad{O(1)}
		\State $it.avanzar()$ \complejidad{O(1)}
	\EndWhile
	\State $it \larr CrearIt(diccACopiar)$ \complejidad{O(1)}
	\State // El iterador vuelve a apuntar al primer elemento, defino todos los elementos del vector en el hash
  	\While {$it.haySiguiente()$} \complejidad{O(diccACopiar.tamanio())}
  		\State $res.accesoRapido[fHash(it.siguienteClave())].$
  		\State   $Definir(c,res.contenedor.Definir(it.siguienteClave(),it.siguienteSignificado()))$ \complejidad{$\theta$(1)}
  		\State $it.avanzar()$ \complejidad{O(1)}
  	\EndWhile
}
{$\theta$(n)}

\algoritmo{iDameIterador}{in a:diccR}{res:itDiccR}{
	\State $res \larr CrearIt(a.contenedor)$ \complejidad{O(1)}
}

\algoritmo{iObtener}{in/out a: diccR , in c : Nat}{res : \beta}{
  	\State $res \larr a.accesoRapido[a.fHash(c)].obtener(c).siguiente()$ \complejidad{$\theta(1)$}
}{$\theta(1)$}

%%%%%%%%%%%%%%%%%%%%%%%%%%%%%%%%%%%%%%%%%%%%%%%%%%%%%%%%%%%%%%%%%%%%%%%%%%%%%%%%%

\subsection{Operaciones privadas}
\algoritmo{fHash}{in a:diccR, in c:nat}{res:Nat}{
	\State $res \larr (c \% a.tamanio())$ \complejidad{O(1)}
}

\subsection{Representación del iterador}

\serc{itDiccRIt}{itDicc(<clave:nat, significado:\beta>)}